\chapter{Localization System Implementation}

\label{Chapter4}

In order to develop and evaluate the system outlined in Chapter \ref{Chapter3}, it is first required to establish a testing environment.
This chapter introduces the implementation of the localization system in a test bed.

\section{System Overview}

The test bed consists of multiple anchor nodes, a mobile node and a computer.

The ANs are commercial WiFi access pints, which are placed in the area of interest and constantly broadcast a beacon signal.

The MN is an Android smartphone. It is used to collect samples from different locations in the area of interest.

The samples collected by the MN are transferred to a computer. The computer is responsible for all the computations. It executes all the algorithms for the room recognition, ranging, weighting and the trilateration.

\begin{figure}[ht]
\centering
\includegraphics[width=\textwidth]{Figures/SystemImplementationOverview}
\decoRule
\caption[Test bed implementation overview]{Overview of the test bed implementation}
\label{fig:localizationSystemOverview}
\end{figure}

In our test bed implementation the smartphone is only used to collect the samples. On the computer these samples are then used as training and evaluation points for the localization system. This way it is possible to try out and empirically compare different parameters of the system under the exact same conditions.

The set-up of this test bed comprises both hardware and software specific configurations for each component. The remainder of this chapter details these configurations.

\section{Hardware}

The hardware for this test bed consists of multiple ANs, a MN and a computer. There are no special requirements for the computer as long as it is able to execute Java code.

For the AN and MN the following hardware was employed:

\paragraph{Anchor Nodes}
The commercial WiFi access points used as anchor nodes are of the model D-Link D-635 and D-2553. They are set-up with a beacon period of 100ms and broadcast on the 2.4 GHz frequency band.

\paragraph{Mobile Node}

The mobile node is an Android smartphone of the model \emph{One Plus One}. It has the following specifications:

\begin{itemize}
\item \textbf{OS:} Android 5.1
\item \textbf{Processor:} 2.5GHz Quad-core CPU
\item \textbf{WiFi module:} Qualcomm WCN3680 802.11ac/FM/BT 4.0 Combo Chip
\item \textbf{Internal sensors:} accelerometer, magnetometer, gyroscope, proximity, ambient light
\item \textbf{Memory:} 3 GB RAM
\end{itemize}

The WiFi module and the magnetometer are used for the sample collection. The magnetometer is reasonably accurate with a resolution of 0.1 $\si\micro\ T$. The off-the-shelf WiFi interface is less reliable because it does only provide fine-grained RSSI values (more fine-grained values would be better) and it does not support channel state information.

\section{Software}

\subsection{Mobile Node}

The mobile node runs an Android application which collects RSSI and magnetic field data.

To collect one sample the application takes the average of five RSSI and magnetometer measurements, each spaced 2 seconds apart. The samples are then saved to a \code{.csv} file on the smartphones internal storage so they can later be transferred to the computer.

Each sample consist of:
\begin{itemize}
\item \textbf{A label} indicating the room or the exact location where the sample was taken.
\item \textbf{A set of RSSI values}, one for each AN.
\item \textbf{The magnetic field strength} in \(\mu\)-Tesla along the devices x,y and z axis.
\end{itemize}

On Android the WiFi module cannot be accessed directly. WiFi scans have to be initiated through the \code{AndroidAPI} and it only supports full scans \cite{brouwers2014incremental}. Full scans take longer so it is only possible to take one RSSI measurement every 1.5 seconds.

Due to the low sampling rate, it is not practical to apply filters to remove noise from the $RSSI$. It is also not possible to access channel state information which could be used to mitigate some of the multi path effects.

\subsection{Computer}

On the computer a few different programs are used to implement the localization system:
\begin{itemize}
\item \textbf{WEKA 3.6} is used for the room recognition classifier.

\item \textbf{Matlab}'s function fit tool is used to train the ranging model.

\item A \textbf{Spreadsheet} program is used to manually calculate the weighting model.


\item The \textbf{trilateration tool} \cite{trilaterationTool} is responsible for the application of the ranging and weighting models and performs the trilateration. It is a small Java application written for this test bed implementation.

\end{itemize}

The implementation of the room recognition system is split into two phases; 
In the \textbf{offline phase} the room recognition, ranging and weighting model are generated. Afterwards, in the \textbf{online phase}, these models are used to predict the location of unknown samples.

Figure \ref{fig:offlineImplementation} gives an overview of this process. In the following the two phases are explained in detail.

\begin{figure}[ht]
\centering
\includegraphics[width=\textwidth]{Figures/Offline_Set-Up}
\decoRule
\caption[Test bed offline implementation]{Diagram of the offline implementation}
\label{fig:offlineImplementation}
\end{figure}

\paragraph{Offline Phase}

\paragraph{Room Recognition Model}
The SVM for the room recognition is trained in WEKA. It generates a multiclass SVM model from the training data set and may use grid search for the parameter selection. The training data set consists of samples containing RSSI \((RSSI_{i})\) and magnetic field \((B_{xyz})\) values labeled with the room number \((R\#)\).

\paragraph{Ranging Model}
For the ranging model the \(\alpha, \beta\) parameters from equation \ref{eqn:non-linear path loss model} need to be determined for each AN. This is done by fitting the equation to the testing data in \emph{Matlab}. The testing data set is a list of RSSI values and the corresponding distance \((D_{i})\) to the AN.

The resulting non-linear regression model is inaccurate with high RSSI values (samples very close to the AN). To account for that the distances for these high values are set by hand.

\paragraph{Weighting Model}

The \emph{Room Weights} are calculated by hand based on equation \ref{eqn: Room Weights} in a spreadsheet program and imported into the trilateration tool.

The \emph{Distance Weights} are calculated by the trilateration tool during run time based on the equation \ref{eqn: distance weights}.

\paragraph{Online Phase}

The online dataset contains samples with RSSI and magnetic field values.

In a first step WEKA is used to predict the room number. The result is handed to the trilateration tool, which applies the ranging and weighting models to determine the predicted distances \((d_{i})\) and calculate the weights \((w_{i})\). It then solves the trilateration problem (equation \ref{eqn: trilateration as optimization problem}) using the Levenberg–Marquardt optimizer from the \code{Apache Commons Math} library. It outputs the predicted position \((x,y)\) to a \code{.csv} file.

To evaluate the system, the online dataset can also contain the actual position \((X,Y)\) where the sample was collected. In this case the trilateration tool will also output the localization error.
