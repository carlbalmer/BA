\chapter{The localization system}

\label{Chapter3}

In chapter \ref{Chapter2} the two most common indoor localization approachers were introduced. In this chapter a localization system is proposed that combines those approaches, using fingerprinting to improve the range-based approach.

\paragraph{}The proposed system adds a room recognition system and new weighting method to the standard range-based approach \note{insert reference}. The room recognition uses fingerprintig with $RSSI$ and magnetic field data to determine the devices current room. The new weightinh method then relies on the information provided by the room recognition to more accurately estimate the ranging error and improve the trilateration accuracy.

\begin{figure}[ht]
\centering
\includegraphics[width=\textwidth]{Figures/proposedApporach}
\decoRule
\caption[The proposed approach]{Block diagram of the proposed system.}
\label{fig:proposedApproach}
\end{figure}

As apparent in the block diagramm (Figure \ref{fig:proposedApproach} the standard range-based approach is not changed but simply extended. Therefore the main focus of this work are the two added components; the proposed room recognition systen and weighting method. In the remainder of this chapter those components are explained in detail.




During localization the room recognition system detects the current room \(\#R\) based on \(RSSI\) and magnetic field data \(M_{x,y,z}\). The weighting model then uses this knowledge to calculate the weights. \red{It may also take \(d_i\) into account.}

This means that during the training phase not only the ranging model but also the fingerprint-map and the weighting model need to be created. Thus to keep the approach practical this creation process should be as simple as possible.

\red{The main focus of this work is this second layer and its components the room recognition and weighting.} 

\subsection{Room Recognition}
The objective is a simple, easy to train system, which predicts the devices current room with high accuracy.

To achieve this I chose to use a fingerprinting method. The main problem with fingerprinting for \red{traditional} localization is that it needs a lot of reference points and for each reference point the exact location has to be measured. This is often very time consuming and therefore impractical.

However for room recognition the RPs only need to be labeled with a room number. In practice this means that\red{, depending on the sampling rate of the sensors,} a large number of RPs can be collected by walking once through each room and "scanning" it with the smartphone. Thus a high density map can be created in in a short time.

The second thing to consider are which observation parameters to use as features in the fingerprinting map. In a best case there should be a high number of attributes with high local variability and they should not require any additional hardware or setup. It makes sense to use the RSSI values as those are already required for the ranging and trilateration. Additionally I chose to use the magnetometer readings as previous research has shown its applicably for fingerprinting-localization.

\note{mention the third problem: how to collect pints and put in a better seaway into the development part}

The fingerprinting map is used to train a multiclass SVM classifier, which is then employed during localization to predict the devices room.
\subsubsection{Development}
\label{overviewRoomRecognition}
I used my apartment as a small test bed to test the feasibility of this proposed room recognition system. The objective was to answer the following questions:
\begin{itemize}
\item \red{What is the best way to gather the RPs for the fingerprinting-map How many RPs are needed and how does their distribution affect the accuracy?}
\item How does the magnetic field data influence the accuracy. Does it improve the accuracy as presumed?
\item What accuracy can be expected form the system. Does it even achieve high enough accuracy?
\end{itemize}

To answer the first question I gathered a number of fingerprinting maps with different properties and evaluated their accuracy.

This experiment concluded that the highest accuracy could be achieved by taking a lot of RPs at the borders of the room (doors, walls between rooms) and only a few in the center.This conclusion is also supported by the theory of SWM as the RPs at the borders end up being the support vectors that define the hyperplane.

Using this technique I was able to achieve a \note{inser percent} accuracy. \red{Which I think schould be enough for this use case.}

I also compared the performance with and without magnetic field data. Including the magnetic field data generally increased the accuracy by \note{inset percent here}

\subsection{Weighting}

The Weighting step is where the final weights are calculated. This is done with a mathematical model, which relates the available information, \(d_i,\#R\) to a set of weights \(w_i\). This model should be created during the training phase.

For the training data the actual distance to the ANs is known. So by comparing the actual and estimated distances for all training samples in a room the ranging error can be calculated. The weights can then be defined as inversely proportional to the average variance of ranging errors in a room. This would lead to a separate set of weights for each room.

The above method does not take \(d_i\) into account. It could be beneficial to combine the room weights with the distance based weights from the existing approach. \(d_i\) could also be use to discredit impossibly large estimations.

\begin{equation}
w_{Ri}=\frac{(\sum_{s=1}^{S_R}{(D_{si}-d_{si})^2})^{-1}}{\sum_{i=1}^{N}{(\sum_{s=1}^{S_R}{(D_{sn}-d_{sn})^2})^{-1}}}
\end{equation}


\subsubsection{Development}
\label{WeightingModelDefinition}

In order to test these hypothesis and determine the best performing weighting model, I experimented with different ways to calculate and combine the room and distance weights.

For this I used the dataset gatered from the third floor of the \red{IAM-Building}. The performance of the different methods was asses under the assumption of 100\% room recognitoin accuracy and then compared with OLS and the distance weights form the existing approach.

\red{From this experiment I found out, that the room weights by themselves already performed quit well. Combining them with the distance weights removes some of the outlines and generally improves the performance. Not considering the outlines in the room weights calculation decreased the performance.}

To conclude the final weighting method is this:

\note{insert formula}



There is room for improvement in the way that the existing approach determines the trilateration weights. These weights are used to correct for the error introduced by the ranging. So to determine the weights one needs to know the ranging error. The existing approach makes the simple assumption, that the ranging error increases with distance to the AN. \red{This is not optimal as it assesses the error in \(d_i\) based only on \(d_i\) itself}.

Supposed that the error is mainly affected by obstructions to the signal (walls, wires etc.). This would mean, that inside one room where there are no major obstructions to the signal the error should be \red{approximately the same}. So it should be possible to get an accurate estimate of the ranging error based on the room the device is currently in, therefore improving the weights and achieving a more accurate trilateration and localization.