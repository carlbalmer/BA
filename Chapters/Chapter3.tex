\chapter{The localization system} % Main chapter title

\label{Chapter3} % Change X to a consecutive number; for referencing this chapter elsewhere, use \ref{ChapterX}

In this chapter I will briefly explain the existing localization approach and then go into detail about my proposed improvements.\note{Better explain why I'm talking about Room recognition and Weighting in detail.}

\section{The existing approach}

The following diagram shows the exiting approach that was used for previous work in the CDS group \cite{JoseMaster}:

\begin{figure}[ht]
\centering
\includegraphics[width=\textwidth]{Figures/existingAproach}
\decoRule
\caption[The existing approach]{Block diagram of the existing approach.}
\label{fig:existingApproach}
\end{figure}

It is a range-based approach and consists of two main phases. In the first phase, the training phase, a small number of training samples are gathered from the building. These samples are value-label pairs with the value being the \(RSSIs\) received from the different ANs and the label the \(x,y\) coordinates of the current location in the building. The samples are gathered with a smartphone. \red{In the case of \cite{josePaper} 20 points were used for a floor with approximate 250m\textsuperscript{2} and eight rooms}. The training samples are then used to train the ranging models explained in chapter \ref{Ranging}.

\red{During localization, the second phase,} RSSI readings are provided by the smartphone (MN). These are subsequently converted into distance estimations \(d_i\)  using the ranging models. Trilateration is then applied to determine the smartphones location. The trilateration is ether performed as OLS without any weights or with weights inversely proportional to the estimated distances.

\section{Proposed Improvements}

There is room for improvement in the way that the existing approach determines the trilateration weights. These weights are used to correct for the error introduced by the ranging. So to determine the weights one needs to know the ranging error. The existing approach makes the simple assumption, that the ranging error increases with distance to the AN. \red{This is not optimal as it assesses the error in \(d_i\) based only on \(d_i\) itself}.

Supposed that the error is mainly affected by obstructions to the signal (walls, wires etc.). This would mean, that inside one room where there are no major obstructions to the signal the error should be \red{approximately the same}. So it should be possible to get an accurate estimate of the ranging error based on the room the device is currently in, therefore improving the weights and achieving a more accurate trilateration and localization.

So I propose a room recognition system based on which the weights for the trilateration are determined.

\noindent The new approach would look like this:

\begin{figure}[ht]
\centering
\includegraphics[width=\textwidth]{Figures/proposedApporach}
\decoRule
\caption[The proposed approach]{Block diagram of the proposed approach.}
\label{fig:proposedApproach}
\end{figure}

The main approach is not changed. It is extended by a \red{second layer}, which is responsible for determining the weights. This layer consist of a room recognition system based on fingerprinting and a weighting model \red{which relates room and weights}.

During localization the room recognition system detects the current room \(\#R\) based on \(RSSI\) and magnetic field data \(M_{x,y,z}\). The weighting model then uses this knowledge to calculate the weights. \red{It may also take \(d_i\) into account.}

This means that during the training phase not only the ranging model but also the fingerprint-map and the weighting model need to be created. Thus to keep the approach practical this creation process should be as simple as possible.

\red{The main focus of this work is this second layer and its components the room recognition and weighting.} 

\subsection{Room Recognition}
The objective is a simple, easy to train system, which predicts the devices current room with high accuracy.

To achieve this I chose to use a fingerprinting method. The main problem with fingerprinting for \red{traditional} localization is that it needs a lot of reference points and for each reference point the exact location has to be measured. This is often very time consuming and therefore impractical.

However for room recognition the RPs only need to be labeled with a room number. In practice this means that\red{, depending on the sampling rate of the sensors,} a large number of RPs can be collected by walking once through each room and "scanning" it with the smartphone. Thus a high density map can be created in in a short time.

The second thing to consider are which observation parameters to use as features in the fingerprinting map. In a best case there should be a high number of attributes with high local variability and they should not require any additional hardware or setup. It makes sense to use the RSSI values as those are already required for the ranging and trilateration. Additionally I chose to use the magnetometer readings as previous research has shown its applicably for fingerprinting-localization.

\note{mention the third problem: how to collect pints and put in a better seaway into the development part}

The fingerprinting map is used to train a multiclass SVM classifier, which is then employed during localization to predict the devices room.
\subsubsection{Development}
\label{overviewRoomRecognition}
I used my apartment as a small test bed to test the feasibility of this proposed room recognition system. The objective was to answer the following questions:
\begin{itemize}
\item \red{What is the best way to gather the RPs for the fingerprinting-map How many RPs are needed and how does their distribution affect the accuracy?}
\item How does the magnetic field data influence the accuracy. Does it improve the accuracy as presumed?
\item What accuracy can be expected form the system. Does it even achieve high enough accuracy?
\end{itemize}

To answer the first question I gathered a number of fingerprinting maps with different properties and evaluated their accuracy.

This experiment concluded that the highest accuracy could be achieved by taking a lot of RPs at the borders of the room (doors, walls between rooms) and only a few in the center.This conclusion is also supported by the theory of SWM as the RPs at the borders end up being the support vectors that define the hyperplane.

Using this technique I was able to achieve a \note{inser percent} accuracy. \red{Which I think schould be enough for this use case.}

I also compared the performance with and without magnetic field data. Including the magnetic field data generally increased the accuracy by \note{inset percent here}

\subsection{Weighting}

The Weighting step is where the final weights are calculated. This is done with a mathematical model, which relates the available information, \(d_i,\#R\) to a set of weights \(w_i\). This model should be created during the training phase.

For the training data the actual distance to the ANs is known. So by comparing the actual and estimated distances for all training samples in a room the ranging error can be calculated. The weights can then be defined as inversely proportional to the average variance of ranging errors in a room. This would lead to a separate set of weights for each room.

The above method does not take \(d_i\) into account. It could be beneficial to combine the room weights with the distance based weights from the existing approach. \(d_i\) could also be use to discredit impossibly large estimations.

\subsubsection{Development}
\label{WeightingModelDefinition}

In order to test these hypothesis and determine the best performing weighting model, I experimented with different ways to calculate and combine the room and distance weights.

For this I used the dataset gatered from the third floor of the \red{IAM-Building}. The performance of the different methods was asses under the assumption of 100\% room recognitoin accuracy and then compared with OLS and the distance weights form the existing approach.

\red{From this experiment I found out, that the room weights by themselves already performed quit well. Combining them with the distance weights removes some of the outlines and generally improves the performance. Not considering the outlines in the room weights calculation decreased the performance.}

To conclude the final weighting method is this:

\note{insert formula}

\section{Technical Implementation}
\label{technicalImplementation}

This is the implementation of the system, that was used for the evaluation. The smartphone is only used to gather the data sets. All the calculations are then done on a computer using Matlab, libSVM and a custom java tool for the trilateration. In the training phase the data-sets are collected and the models are trained. The ranging and weighting models are then imported into the trilateration tool.

For localization or testing the testing data is first given to the libSVM predict function together with the room recognition model to predict the room number. Next the testing data and room predictions are handed to the trilateration tool. It applies both the ranging and weighting models and then solves the trilateration problem with a Levenberg-Marquardt optimizer. Finally it calculates the accuracy and prints everything to a file.

Below the process for the data collection and model generation is covered in mode detail:

\subsection{Training Phase}
\subsubsection{Collecting the data-sets}

The data-sets are collected with the smartphone (Model: OnePus One) using a small application written for this purpose. The samples consist of a RSSI value for each AN, the magnetic field strength in \(\mu\)-Tesla along the devices x,y and z axis and a label. They are taken by holding the smartphone horizontally approximately one meter above the ground and averaging over five 
consecutive RSSI and magnetometer measurements.

All samples have to be taken with the smartphone pointing in the same cardinal direction, because the magnetometer readings are dependent upon the orientation of the device.

\note{Maybe talk about the sampling rate}

Two data-sets are collected:

\begin{itemize}
\item The fingerprinting map

It contains a high number of samples labeled with a room number. They are collected using the procedure detailed in section \ref{overviewRoomRecognition}.
\note{detail procedure in section above}

\item Ranging data

This contains a smaller number of points which are labeled with \(x,y\)-coordinates. They should be evenly distributed over the area of the test bed.
\end{itemize}

\subsubsection{Training the Models}

\subsubsection{Room Recognition}
The SVM is trained with an automated script provided by the libSVM library. It generates a multiclass SVM model from the fingerprinting map \red{using cross validation to determine the best value for C}.

\subsubsection{Ranging}
For the ranging model the \(\alpha, \beta\) parameters from equation \ref{eqn:non-linear path loss model} need to be determined for each AN. So in a spreadsheet the distance between the AN and the samples is calculated from the ranging data. This results in a list of distances and RSSI values. Matlabs function fit tool is then used to find the \(\alpha, \beta\)-values that best fit this data.
\note{Insert example image here}

As apparent in the example above, this non-linear path loss model is inaccurate with high RSSI values. To account for that the distances for these values are set by hand.

So at the end the model looks like this:

\note{insert formula here}

\subsubsection{Weighting}

The room weights are determined by hand in a spreadsheet based on the formal definition given in section \ref{WeightingModelDefinition}
\note{talk about implementation of dinamic factor in java}