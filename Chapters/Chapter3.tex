\chapter{The localization approach} % Main chapter title

\label{Chapter3} % Change X to a consecutive number; for referencing this chapter elsewhere, use \ref{ChapterX}

In this chapter I will briefly explain the existing localization approach and then go into detail about my proposed improvements.

\section{The existing approach}

The following diagram shows the exiting approach that was used for previous work in the CDS group \cite{josePaper}:

\begin{figure}[ht]
\centering
\includegraphics[width=\textwidth]{Figures/existingAproach}
\decoRule
\caption[The existing approach]{Block diagram of the existing approach.}
\label{fig:existingApproach}
\end{figure}

It is a range-based approach and consists of two main phases. In the first phase, the training phase, a small number of training samples are gathered from the building. These samples are value-label pairs with the value being the \(RSSIs\) received from the different ANs and the label the \(x,y\) coordinates of the current location in the building. The samples are gathered with a smartphone. \red{In the case of \cite{josePaper} 20 points were used for a floor with approximate 250m\textsuperscript{2} and eight rooms}. The training samples are then used to train the ranging models explained in chapter \ref{Ranging}.

\red{During localization, the second phase,} RSSI readings are provided by the smartphone (MN). These are subsequently converted into distance estimations \(d_i\)  using the ranging models. Trilateration is then applied to determine the smartphones location. The trilateration is ether performed as OLS without any weights or with weights inversely proportional to the estimated distances.

\section{Proposed Improvements}

There is room for improvement in the way that the existing approach determines the trilateration weights. These weights are used to correct for the error introduced by the ranging. So to determine the weights one needs to know the ranging error. The existing approach makes the simple assumption, that the ranging error increases with distance to the AN. \red{This is not optimal as it assesses the error in \(d_i\) based only on \(d_i\) itself}.

Supposed that the error is mainly affected by obstructions to the signal (walls, wires etc.). This would mean, that inside one room where there are no major obstructions to the signal the error should be \red{approximately the same}. So it should be possible to get an accurate estimate of the ranging error based on the room the device is currently in, therefore improving the weights and achieving a more accurate trilateration and localization.

So I propose a room recognition system based on which the weights for the trilateration are determined.

\noindent The new approach would look like this:

\begin{figure}[ht]
\centering
\includegraphics[width=\textwidth]{Figures/proposedApporach}
\decoRule
\caption[The proposed approach]{Block diagram of the proposed approach.}
\label{fig:proposedApproach}
\end{figure}

The main approach is not changed. It is extended by a \red{second layer}, which is responsible for determining the weights. This layer consist of a room recognition system based on fingerprinting and a weighting model \red{which relates room and weights}.

During localization the room recognition system detects the current room \(\#R\) based on \(RSSI\) and magnetic field data \(M_{x,y,z}\). The weighting model then uses this knowledge to calculate the weights. \red{It may also take \(d_i\) into account.}

This means that during the training phase not only the ranging model but also the fingerprint-map and the weighting model need to be created. Thus to keep the approach practical this creation process should be as simple as possible.

\subsection{Room Recognition}
The objective is a simple, easy to train system, which predicts the devices current room with high accuracy.

To achieve this I chose to use a fingerprinting method. The main problem with fingerprinting for \red{traditional} localization is that it needs a lot of reference points and for each reference point the exact location has to be measured. This is often very time consuming and therefore impractical.

However for room recognition the RPs only need to be labeled with a room number. In practice this means that\red{, depending on the sampling rate of the sensors,} a large number of RPs can be collected by walking once through each room and "scanning" it with the smartphone. Thus a high density map can be created in in a short time.

The second thing to consider are which observation parameters to use as features in the fingerprinting map. In a best case there should be a high number of attributes with high local variability and they should not require any additional hardware or setup. It makes sense to use the RSSI values as those are already required for the ranging and trilateration. Additionally I chose use the magnetometer readings as previous research has shown its applicably for fingerprinting-localization.

The fingerprinting map is used to train a multiclass SVM classifier, which is then employed during localization to predict the devices room.

\subsection{Weighting}

The Weighting step is where the final weights are calculated. This is done with a mathematical model, which relates the available information, \(d_i,\#R\) to a set of weights \(w_i\). This model should be created during the training phase. There are an number of possible methods to do this:

\begin{itemize}
\item Define the weights as inversely proportional to the average variance of ranging errors in a room.

For the training data the actual distance to the ANs is known. So by comparing the actual and estimated distances for all training samples in a room the ranging error can be predicted. This would lead to a separate set of weights for each room.
\item Define the weights based on the residuals form OLS

This would mean running the trilateration one as OSL with the training data and then choosing a set of weights so that the residuals are minimized for all training samples in a room. This again leads to a separate set of weights for each room.
\item Incorporating the distance estimation into the above methods.

The two above methods lead to a separate set of weights for each room. They do not take \(d_i\) into account. It could be beneficial to combine the room weights from the above method with the distance based weights from the existing approach. It could also be use to discredit impossibly large estimations.
\item Calculating optimal weights based on the training data

The location of a training sample is known. So it should be possible to calculate the weights which lead to the most accurate localization. 
\end{itemize}

In the next chapter I will investigate which of these methods actually performs best.